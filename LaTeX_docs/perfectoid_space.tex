\documentclass{amsart}
\usepackage{amsmath}
\usepackage{amsthm}
%\usepackage{a4wide}
%\usepackage{enumerate}

\newcommand{\R}{\mathbb{R}}
\newcommand{\Z}{\mathbb{Z}}

\theoremstyle{plain}
\newtheorem{theorem}{Theorem}
\newtheorem{lemma}[theorem]{Lemma}
\newtheorem{corollary}[theorem]{Corollary}
\newtheorem{proposition}[theorem]{Proposition}
\theoremstyle{remark}
\newtheorem{remark}[theorem]{Remark}
\newtheorem*{remarkn}{Remark}

\begin{document}

\section{Notes on {\tt perfectoid.lean}}

This short file gives the definitions of a perfectoid ring and a perfectoid space. Of course all the heavy lifting (definitions such as {\tt is\_complete}, {\tt is\_uniform}, {\tt is\_pseudo\_uniformizer}, {\tt Tate\_ring}, {\tt Huber\_pair} and, most importantly, {\tt adic\_space}) is done elsewhere.

\end{document}
